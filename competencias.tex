% competencias.tex


\subsection*{2.1 Genéricas}
% Seleccione máximo 3 de las 12 competencias genéricas que se listan a continuación y borre las demás.
% Si quiere ponerlo en 1 columna comente las líneas de \begin{multicols}{2} y \end{multicols}.

    \begin{enumerate}[label={}]
        \item[3.] Trabaja en equipo.
        \item[4.] Resuelve problemas de manera creativa y efectiva.
        \item[5.] Utiliza adecuadamente la tecnología.
    \end{enumerate}




\subsection*{2.2 Específicas}
% Defina las competencias específicas del curso. Para cada competencia, liste los saberes conceptuales, procedimentales y actitudinales que se desarrollarán en el curso.
% Si su curso tiene más de 3 competencias específicas, copie y pegue el bloque de código de la competencia las veces que sea necesario.
\competencia{Aplica conocimientos básicos de estadísticas para, luego de comprender el problema planteado, seleccione la mejor técnica a utilizar para resolverlo.}{
    \item Comprensión de las variables de estudio.
    \item Estadística descriptiva
    \item Análisis exploratorio.
    \item Reconocimiento de la variable respuesta.
}{
    \item Realiza un análisis de las variables del conjunto de datos.
    \item Reconoce la variable respuesta y las variables que aportan más al modelo que se desea obtener.
    \item Utiliza la estadística descriptiva para resumir las características principales de un conjunto de datos.
    \item Emplea técnicas de análisis exploratorio para identificar patrones, detectar anomalías y probar hipótesis con la ayuda de herramientas computacionales.
}{
    \item Demuestra concentración, paciencia y persistencia en las actividades que realiza.
    \item Manifiesta disciplina y orden en la solución de problemas.
}


\competencia{Analiza el problema a resolver para aplicar la técnica de aprendizaje automático más adecuada para su resolución.}{
    \item Aplicaciones del aprendizaje automático para la resolución de problemas.
    \item Tipos de aprendizaje automático (supervisado, no supervisado, semi-supervisado y por refuerzo )
    \item Construcción de modelos de aprendizaje automático.
    \item Indicadores de rendimiento de modelos de aprendizaje automático: exploratorios y predictivos.
    \begin{itemize}
        \item Métricas de evaluación para modelos de clasificación.
        \item Métricas de evaluación para modelos de regresión.
        \item Relación sesgo-varianza.
    \end{itemize}
    \item Técnicas para evitar sobreajuste: 
    \begin{itemize}
        \item Validación Cruzada
        \item Remuestreo (Bootstrap)
    \end{itemize}
   
}{
    \item Determina la técnica de aprendizaje automático a utilizar dada la situación planteada.
    \item Selecciona los modelos y algoritmos adecuados para resolver el problema.
    \item Plantea indicadores de rendimiento de los distintos algoritmos estudiados para seleccionar el mejor modelo.
}{
    \item Demuestra concentración, paciencia y persistencia en las actividades que realiza.
    \item Manifiesta disciplina y orden en la solución de problemas.
    \item Posee iniciativa, motivación y actitud positiva para enfrentar los retos que se le presenten.
    \item Tiene disposición para comunicarse efectivamente y asumir responsabilidades en un equipo de trabajo.
}

\competencia{Implementa algoritmos de aprendizaje automático utilizando modelos predictivos y clasificatorios según sea el caso.}{
    \item Algoritmos de aprendizaje no supervisados: 
    \item \begin{itemize}
        \item Clustering: K-means, clustering jerárquico, Mixture of Gaussians.
        \item Reglas de asociación: Apriori
        \item Reducción de dimensionalidad: PCA
        \item Otros algoritmos no supervisados: SVD, t-SNE, UMAP, ICA 
    \end{itemize} 
    \item Algoritmos de aprendizaje supervisados:
    \begin{itemize}
        \item Regresión lineal
        \item Regresión logística.
        \item Árboles de decisión y Random Forest.
        \item Máquinas de vectores de soporte (SVM).
        \item Redes neuronales artificiales.
        \item Bayes ingenuo.
        \item K-vecinos más cercanos (KNN).
    \end{itemize}
    \item Algoritmos de aprendizaje semisupervisados:   
    \begin{itemize}
        \item Self-training
        \item Co-training
        \item Propagación
        \item SVM semi supervisado
        \item Generativos
        \item k-means con restricciones
        \item Laplacian SVM.
    \end{itemize} 
}{
    \item Aplica algoritmos de minería de datos en data sets de la vida real.
    \item Utiliza las principales características de cada algoritmo para identificar qué tipo de algoritmo debe de utilizarse en cada situación.
    \item Diseña programas y soluciones donde utilizan uno o varios algoritmos de aprendizaje automático para resolver un problema.
    \item Determina la eficiencia del algoritmo para el problema planteado
}{
 \item Demuestra concentración, paciencia y persistencia en las actividades que realiza.
    \item Manifiesta disciplina y orden en la solución de problemas.
    \item Posee iniciativa, motivación y actitud positiva para enfrentar los retos que se le presenten.
    \item Tiene disposición para comunicarse efectivamente y asumir responsabilidades en un equipo de trabajo.
}


\subsection*{2.2.1 Específicas de la carrera}
%En la primera columna hay que poner las competencias específicas de la carrera y en la segunda columna que competencia del curso ayudan a obtenerla.
\begin{table}[H]
    \centering
    \setlength{\extrarowheight}{5pt}
    \begin{tabularx}{\textwidth}{|X|X|X|X|}
        \hline
         \multicolumn{1}{|c|}{\textbf{Competencia Específica de la Carrera}} & 
         \multicolumn{1}{c|}{\textbf{Subcompetencias desarrolladas en el curso}} \\ \hline 
         CE12. Desarrollar algoritmos inteligentes con la representación y mecanismo de razonamientos adecuados en contextos específicos.& Implementa algoritmos de aprendizaje automático utilizando modelos predictivos y clasificatorios según sea el caso. \\ \hline 
         CE13. Evaluar los algoritmos inteligentes más adecuados para problemas específicos.& Analiza el problema a resolver para aplicar la técnica de aprendizaje automático más adecuada para su resolución. \\ \hline 
         CE16. Analizar datos empleando algoritmos y herramientas de aprendizaje de máquina y minería de datos.& Aplica conocimientos básicos de estadísticas para, luego de comprender el problema planteado, seleccione la mejor técnica a utilizar para resolverlo. \\ \hline 
    \end{tabularx}
\end{table}
