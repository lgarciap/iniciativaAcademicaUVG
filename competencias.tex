% competencias.tex

\subsection*{3.1 Genéricas}
\begin{enumerate}[label=\Alph*.]
    \item Piensa de manera crítica y analítica.
    \item Trabaja en equipo.
    \item Resuelve problemas de manera creativa y efectiva.
    \item Actúa éticamente.
    \item Aprende a aprender autónomamente.
\end{enumerate}

\subsection*{3.2 Específicas}
\begin{enumerate}[label=\Alph*.]
    \item Competencia 1.
    \item Competencia 2.
\end{enumerate}

%En la primera columna hay que poner las competencias específicas de la carrera y en la segunda columna que competencia del curso ayudan a obtenerla.
\begin{table}[h!]
    \centering
    \setlength{\extrarowheight}{5pt}
    \begin{tabularx}{\textwidth}{|X|X|X|X|}
        \hline
         \multicolumn{1}{|c|}{\textbf{Competencia Específica de la Carrera}} & 
         \multicolumn{1}{c|}{\textbf{Subcompetencias desarrolladas en el curso}} \\ \hline 
         CE15: Organizar y depurar los datos provenientes de fuentes diversas, estructuradas y no estructuradas y/o en grandes volúmenes (“big data”) para que puedan ser analizados.& Competencia 1 \\ \hline 
         CE16: Analizar datos empleando algoritmos y herramientas de aprendizaje de máquina yminería de datos.& Competencia 2 \\ \hline 
    \end{tabularx}
\end{table}

\newpage
\competencia{Competencia 1}{
    \item Saber conceptual 1.
    \item Saber conceptual 2.
    \item Saber conceptual n.
}{
    \item Saber procedimental 1.
    \item Saber procedimental 2.
    \item Saber procedimental n.
}{
    \item Saber actitudinal 1.
    \item Saber actitudinal 2.
    \item Saber actitudinal n.
}

\newpage
\competencia{Competencia 2}{
    \item Saber conceptual 1.
    \item Saber conceptual 2.
    \item Saber conceptual n.
}{
    \item Saber procedimental 1.
    \item Saber procedimental 2.
    \item Saber procedimental n.
}{
    \item Saber actitudinal 1.
    \item Saber actitudinal 2.
    \item Saber actitudinal n.
}


