% estructuras.tex

\newcommand{\identificacion}{
    \begin{table}[ht]
        \centering
        \setlength{\extrarowheight}{5pt}
        \begin{tabularx}{\textwidth}{|X|X|X|X|}
            \hline
            \textbf{Nombre} & \multicolumn{3}{l|}{\nombreCurso} \\
            \hline
            \textbf{Código} & \multicolumn{3}{l|}{\codigoCurso} \\
            \hline
            \textbf{Créditos} & \multicolumn{3}{l|}{\creditos} \\
            \hline
            \textbf{Año} & {\anio} 
            & \textbf{Ciclo Académico} & {\ciclo} \\
            \hline
            \textbf{Requisitos} & 
            %Si no hay requisitos solo poner \variosItems
            \multicolumn{3}{l|}{\requisitos} \\
            \hline
            \textbf{Modalidad} & \multicolumn{3}{l|}{\modalidades} \\
            \hline
            \textbf{Elaborado} & {\fechaCreacion} 
            & \textbf{Modificado} & {\fechaModificacion} \\
            \hline
        \end{tabularx}
    \end{table}
}

\newcommand{\prereqlist}{}
\newcommand{\addprereq}[1]{%
  \ifx\prereqlist\empty
    \def\prereqlist{#1}%
  \else
    \edef\prereqlist{\prereqlist, #1}%
  \fi
}

\NewDocumentCommand{\variosItems}{O{Ninguno}}{%
  \ifx\relax#1\relax
    Ninguno%
  \else
    \def\prereqlist{}%
    \forcsvlist{\addprereq}{#1}%
    \prereqlist
  \fi
}

\newcommand{\competencia}[4]{
    \begin{table}[htbp!]
        \centering
        \begin{tabularx}{\textwidth}{|X|}
            \hline
            \textbf{Competencia:} #1 \\  % Title
            \hline
        \end{tabularx}
        \begin{tabularx}{\textwidth}{|X|p{5cm}|p{5cm}|}
            \multicolumn{1}{|c|}{\textbf{Saberes conceptuales}} & \multicolumn{1}{c|}{\textbf{Saberes procedimentales}} & \multicolumn{1}{c|}{\textbf{Saberes actitudinales}} \\
            \hline
            \begin{itemize}[noitemsep, topsep=0pt, left=0pt, labelsep=5pt, partopsep=0pt]
                \fontsize{10pt}{11pt}\selectfont
                #2
            \end{itemize} 
            & 
            \begin{itemize}[noitemsep, topsep=0pt, left=0pt, labelsep=5pt, partopsep=0pt]
                \fontsize{10pt}{11pt}\selectfont
                #3
            \end{itemize} 
            & 
            \begin{itemize}[noitemsep, topsep=0pt, left=0pt, labelsep=5pt, partopsep=0pt]
                \fontsize{10pt}{11pt}\selectfont
                #4
            \end{itemize} 
            \\
            \hline
        \end{tabularx}
    \end{table}  
}

