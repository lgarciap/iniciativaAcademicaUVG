% !TeX program = xelatex
\documentclass[letter,11pt]{article}

\usepackage[utf8]{inputenc}
\usepackage[spanish]{babel}
\usepackage{enumitem}
\usepackage{hyperref}
\usepackage{fancyhdr}
\usepackage{array}
\usepackage{booktabs}
\usepackage{tabularx}
\usepackage{geometry}
\usepackage{fontspec}
\usepackage{graphicx}
\usepackage{float}
\usepackage{multicol}

\defaultfontfeatures{Ligatures=TeX}
\setmainfont{Calibri}
\geometry{left=2.5cm, right=2.5cm, top=5cm, headsep=2.5cm}
\pagestyle{fancy}
\setlist[enumerate,1]{topsep=0pt, partopsep=0pt, itemsep=0pt, parsep=0pt}

\author{Lynette García, Erick Marroquín}

% variables.tex

% Información general
% Sección de identificación
\newcommand{\facultad}{Facultad de Ingeniería}
\newcommand{\departamento}{Ciencia de la Computación y Tecnologías de la información}
\newcommand{\nombreCurso}{Nombre Curso}
\newcommand{\codigoCurso}{cc1234}
\newcommand{\creditos}{4}
\newcommand{\anio}{2025}
\newcommand{\ciclo}{2}
\newcommand{\teoria}{2}
\newcommand{\practica}{2}
\newcommand{\total}{2}
\newcommand{\horario}{Lunes y Miércoles 8:00 - 10:00}
\newcommand{\fechaCreacion}{1/7/2024}
\newcommand{\fechaModificacion}{1/06/2025}

\newcommand{\descripcion}{
Descripción del curso
}
\newcommand{\requisitos}{\variosItems[Requisito 1, Requisito 2, Requisito n ]}
\newcommand{\modalidades}{\variosItems[Modalidad 1, Modalidad 2, Modalidad n]}


% Metodologías
% Defina las metodologías activas que se utilizarán en el curso. Para marcar una metodología, 
%ponga una X entre los corchetes antes del texto de la metodología.

\newcommand{\opc}[2][]{\makebox[\linewidth]{#2\hfill(\,#1\,)}}
\newcommand{\metodologiasActivas}{%
  \setlength{\tabcolsep}{8pt}%
  \renewcommand{\arraystretch}{1.2}%
  \noindent\begin{tabularx}{\textwidth}{|X|X|}
    \hline
    \multicolumn{2}{|l|}{\textbf{METODOLOGÍAS ACTIVAS SUGERIDAS (Marque con una X)}} \\ \hline
    % Ponga [X] antes de la llave para marcar la casilla que desee
    %Ejemplo: \opc[X]{Aprendizaje basado en proyectos}
    \opc{Aprendizaje basado en problemas} & \opc{Aprendizaje basado en proyectos} \\ \hline
    \opc{Aprendizaje basado en estudio de casos} & \opc{Aprendizaje basado en investigación} \\ \hline
    \opc{Aprendizaje situado (in situ)} & \opc{Aprendizaje por descubrimiento o heurístico} \\ \hline
    \opc{Autoaprendizaje} & \opc{Aprendizaje entre pares} \\ \hline
    \opc{Aprendizaje basado en clase invertida} & \opc{Aprendizaje basado en retos} \\ \hline
    \opc{Aprendizaje basado en gamificación} & \opc{Aprendizaje basado en design thinking} \\ \hline
    \multicolumn{2}{|p{15cm}|}{\textbf{Otra (especifique):}} \\ \hline

  \end{tabularx}%
}



% Bibliografía
% Tomar en cuenta Bibliografía que se encuentra en la Biblioteca UVG y publicaciones o investigaciones realizadas en UVG. 

\newcommand{\bibliografia}{
    \noindent\setlength{\fboxsep}{10pt}\fbox{%
    \parbox{\textwidth}{%
    \begin{enumerate}[label=\Alph*.]
        \item Canvas: \url{https://uvg.instructure.com} Curso: CC2005
        \item Codecademy. (n.d.). Python course catalog. Codecademy. \url{https://www.codecademy.com/catalog/language/python}
        \item Downey, Allen. \textit{Think Python: How to Think Like a Computer Scientist}. Learning with Python. ISBN 13:9780521898119. \url{http://www.thinkpython.com}. Versión electrónica 3
        \item Gonzáles Duque, Raúl. \textit{Python para todos}. \url{http://mundogeek.net/tutorial-python/}
        \item Rice, John K. \& Rice. John R. \textit{Introduction to Computer Science}. Holt, Rinehart and Winston Inc. SBN: 03-067525-1.
        \item Pandas Development Team. (n.d.). pandas documentation. pandas Documentation. \url{https://pandas.pydata.org/docs/}
        \item Streamlit. (n.d.). Cheat sheet. Streamlit Documentation. \url{https://docs.streamlit.io/develop/quick-reference/cheat-sheet}
        \item Universidad del Valle de Guatemala. (n.d.). Reglamento UVG-13. \url{https://res.cloudinary.com/webuvg/image/upload/v1541189810/WEB/Nosotros/reglamentos/Reg-uvg-13.pdf}
    \end{enumerate}
     }%
  }%
}

\newcommand{\politicas}{
  \textbf{ Acerca de exámenes parciales, exámenes cortos, comprobaciones de lectura (y cual-quier otra evaluación presencial en clase que aplique):}
    \begin{itemize}
      \item Se programarán durante los períodos de clase.
      \item En el caso de exámenes parciales se entregarán hojas a los estudiantes para que, a mano, escriban en ellas las respuestas y soluciones requeridas.
      \item En el caso de exámenes cortos y comprobaciones de lectura, los estudiantes deben contar con hojas de papel para responder a mano las respuestas.
      \item No está permitido que el estudiante se levante de su escritorio o mesa ni que hable durante la realización de este tipo de evaluaciones, salvo que por su naturaleza sean en grupo.
      \item Para todo tipo de exámenes no está permitido que el estudiante cuente con dispositi-vos electrónicos en su área de trabajo ni dentro de su ropa o en su cuerpo. Esta dispo-sición incluye: teléfonos celulares, relojes inteligentes, audífonos de cualquier tipo, an-teojos inteligentes, bocinas, tablets y laptops o cualquier otro dispositivo electrónico a través del cual se pueda consultar información o comunicarse de cualquier forma con otras personas.
      \item En caso se observe que un estudiante tiene un dispositivo electrónico en su área de trabajo, ropa o cuerpo durante la evaluación; se le llamará la atención inmediatamente y la evaluación no se calificará. Para el caso de tablets y laptops, se exceptúan los ca-sos en los que se es indispensable el uso de software, previamente autorizado por el docente.
      \item Cualquier estudiante que sea sorprendido copiando en un examen o presentando un trabajo que no sea propio (entiéndase PLAGIO) será sujeto a las medidas disciplinarias que contemple el reglamento de la Universidad.
      \item Si un estudiante se encuentra inconforme con la nota obtenida en una evaluación, po-drá solicitar al docente una revisión dentro de un periodo de 5 días hábiles después de publicada la nota. La evaluación no debe presentar ninguna señal de alteración, y el caso de ser resuelta a mano, debe estar escrita con lapicero sin tachones.
    \end{itemize}
    \textbf{ Acerca del uso de Inteligencia Artificial:}
    \begin{itemize}
      \item Los estudiantes podrán hacer uso de inteligencia artificial, siempre y cuando la utilicen de forma ética y responsable, con juicio crítico, en los casos en que amerita ser utili-zada y que sea para contribuir en su aprendizaje y desarrollo de competencias, de acuerdo con los lineamientos que establezca el docente y citando su uso de la forma correcta en sus informes.
      \item En ningún caso podrán presentar como propia información obtenida a través de esta tecnología. Es importante que los estudiantes revisen las recomendaciones al res-pecto en estos documentos que se encuentran en la página web institucional: 
      \begin{itemize}[label=-]
        \item \href{https://res.cloudinary.com/webuvg/image/upload/v1737393102/WEB/Nosotros/reglamentos/2025/guia-ia-estudiantes.pdf}{Guía para el uso ético de la inteligencia artificial}.
         \item \href{https://res.cloudinary.com/webuvg/image/upload/v1737393102/WEB/Nosotros/reglamentos/2025/politica-ia-uvg.pdf}{Guía rápida para el uso ético y responsable de la inteligencia artificial}.
      \end{itemize}
    \end{itemize}
}
% estructuras.tex

\newcommand{\identificacion}{
    \begin{table}[ht]
        \centering
        \setlength{\extrarowheight}{5pt}
        \begin{tabularx}{\textwidth}{|X|X|X|X|}
            \hline
            \textbf{Nombre} & \multicolumn{3}{l|}{\nombreCurso} \\
            \hline
            \textbf{Código} & \multicolumn{3}{l|}{\codigoCurso} \\
            \hline
            \textbf{Créditos} & \multicolumn{3}{l|}{\creditos} \\
            \hline
            \textbf{Año} & {\anio} 
            & \textbf{Ciclo Académico} & {\ciclo} \\
            \hline
            \textbf{Requisitos} & 
            %Si no hay requisitos solo poner \variosItems
            \multicolumn{3}{l|}{\requisitos} \\
            \hline
            \textbf{Modalidad} & \multicolumn{3}{l|}{\modalidades} \\
            \hline
            \textbf{Elaborado} & {\fechaCreacion} 
            & \textbf{Modificado} & {\fechaModificacion} \\
            \hline
        \end{tabularx}
    \end{table}
}

\newcommand{\prereqlist}{}
\newcommand{\addprereq}[1]{%
  \ifx\prereqlist\empty
    \def\prereqlist{#1}%
  \else
    \edef\prereqlist{\prereqlist, #1}%
  \fi
}

\NewDocumentCommand{\variosItems}{O{Ninguno}}{%
  \ifx\relax#1\relax
    Ninguno%
  \else
    \def\prereqlist{}%
    \forcsvlist{\addprereq}{#1}%
    \prereqlist
  \fi
}

\newcommand{\competencia}[4]{
    \begin{table}[htbp!]
        \centering
        \begin{tabularx}{\textwidth}{|X|}
            \hline
            \textbf{Competencia:} #1 \\  % Title
            \hline
        \end{tabularx}
        \begin{tabularx}{\textwidth}{|X|p{5cm}|p{5cm}|}
            \multicolumn{1}{|c|}{\textbf{Saberes conceptuales}} & \multicolumn{1}{c|}{\textbf{Saberes procedimentales}} & \multicolumn{1}{c|}{\textbf{Saberes actitudinales}} \\
            \hline
            \begin{itemize}[noitemsep, topsep=0pt, left=0pt, labelsep=5pt, partopsep=0pt]
                \fontsize{10pt}{11pt}\selectfont
                #2
            \end{itemize} 
            & 
            \begin{itemize}[noitemsep, topsep=0pt, left=0pt, labelsep=5pt, partopsep=0pt]
                \fontsize{10pt}{11pt}\selectfont
                #3
            \end{itemize} 
            & 
            \begin{itemize}[noitemsep, topsep=0pt, left=0pt, labelsep=5pt, partopsep=0pt]
                \fontsize{10pt}{11pt}\selectfont
                #4
            \end{itemize} 
            \\
            \hline
        \end{tabularx}
    \end{table}  
}



% encabezado.tex

\fancyhf{}
\fancyhead[L]{%
    \raisebox{0.6cm}{%
        \begin{tabular}{@{}l@{}}
            \includegraphics[width=2cm,height=2cm]{imagenes/logo_uvg.png}
        \end{tabular}%
    }%
    \quad
    \begin{tabular}[b]{@{}l@{}}
        Universidad del Valle de Guatemala \\
        {\facultad} \\
        {\departamento} \\
        \textbf{{\codigoCurso} - {\nombreCurso}}
    \end{tabular}%
}
 \fancyhead[R]{Ciclo {\ciclo}, {\anio}}
\fancyfoot[R]{Página \thepage}
\renewcommand{\headrulewidth}{0pt}


\begin{document}

    % ---------------------------------------------------
    % Llene los valores de las variables en la sección de variables.tex
    % ---------------------------------------------------

    \section*{1. Identificación}
    \identificacion

    % ---------------------------------------------------
    % Seleccione máximo 3 de las 12 competencias genéricas y borre las demás en la subsección de competencias genéricas en el archivo competencias.tex
    % Defina las competencias específicas del curso en la subsección de competencias específicas en el archivo competencias.tex
    % ---------------------------------------------------
   \section*{2. Competencias a desarrollar}
    % competencias.tex

\subsection*{2.1 Genéricas}
\begin{enumerate}[label=\Alph*.]
    \item Piensa de manera crítica y analítica.
    \item Trabaja en equipo.
    \item Resuelve problemas de manera creativa y efectiva.
    \item Actúa éticamente.
    \item Aprende a aprender autónomamente.
\end{enumerate}

\subsection*{2.2 Específicas}

\newpage
\competencia{Competencia 1}{
    \item Saber conceptual 1.
    \item Saber conceptual 2.
    \item Saber conceptual n.
}{
    \item Saber procedimental 1.
    \item Saber procedimental 2.
    \item Saber procedimental n.
}{
    \item Saber actitudinal 1.
    \item Saber actitudinal 2.
    \item Saber actitudinal n.
}


\competencia{Competencia 2}{
    \item Saber conceptual 1.
    \item Saber conceptual 2.
    \item Saber conceptual n.
}{
    \item Saber procedimental 1.
    \item Saber procedimental 2.
    \item Saber procedimental n.
}{
    \item Saber actitudinal 1.
    \item Saber actitudinal 2.
    \item Saber actitudinal n.
}

\competencia{Competencia 3}{
    \item Saber conceptual 1.
    \item Saber conceptual 2.
    \item Saber conceptual n.
}{
    \item Saber procedimental 1.
    \item Saber procedimental 2.
    \item Saber procedimental n.
}{
    \item Saber actitudinal 1.
    \item Saber actitudinal 2.
    \item Saber actitudinal n.
}




    % ---------------------------------------------------
    % Marque con una X las metodologías activas que se utilizarán en el curso en la sección de variables.tex
    % Si desea agregar otra metodología, escríbala en el espacio provisto.
    % Esto lo puede hacer en el archivo variables.tex
    % ---------------------------------------------------
    \section*{3. Metodologías activas para la enseñanza y el aprendizaje}
    \metodologiasActivas

    % ---------------------------------------------------
    % Determine los criterios de evaluación del curso en el archivo evaluacion.tex
    % ---------------------------------------------------

    \section*{4. Evaluación}
    % evaluacion
\begin{table}[H]
    \centering
    \setlength{\extrarowheight}{5pt}
    \begin{tabularx}{\textwidth}{|X|p{4cm}|c|}
    \hline
    \multicolumn{3}{|l|}{$\bullet$ \textbf{Competencia: }}\\
    \hline
    \textbf{Criterios de evaluación} & \textbf{Productos y desempeños} & \textbf{\%} \\
    \hline
    &  &   \\ 
    \hline
    &  &  \\
    \hline
    &  &   \\
    \hline
    &  &   \\
    \hline
    &  &   \\
    \hline
    \multicolumn{2}{|r|}{\textbf{Total Competencia}} & \textbf{100} \\
    \hline
    \end{tabularx}
    \label{tab:evaluacion}
\end{table}

    
    % ---------------------------------------------------
    % Determine el cronograma del curso en el archivo cronograma.tex
    % ---------------------------------------------------

    \section*{5. Cronograma}
    \begin{table}[H]
    \centering
    \setlength{\extrarowheight}{5pt}
    \begin{tabularx}{\textwidth}{|p{5cm}|p{1.3cm}|X|}
    \hline
    \textbf{Competencia} & \textbf{Semana} & \textbf{Procedimiento de Evaluación} \\
    \hline
    Competencia 1 &  1& Observación y Diálogo participativo  \\ 
    \hline
    Competencia 1 &  2&  Foro de discusión sobre aplicaciones de la minería de datos y el aprendizaje automático\\
    \hline
    Competencia 1 &  3&  Laboratorio 1. Análisis Exploratorio \\
    \hline
    Competencias 1 y 2&  4& Avances 1 del proyecto. Comprensión del problema, variables disponibles  \\
    \hline
    Competencias 1 y 2&  5& Tarea 2. Otros algoritmos de aprendizaje no supervisado. Laboratorio 2. aprendizaje no supervisado  \\  
    \hline
    Competencias 1 y 2&  6& Entrega 1 del proyecto. Análisis exploratorio y clustering  \\ 
    \hline
    Competencias 2 y 3&  7 & Observación y Diálogo participativo \\
    \hline
    Competencias 2 y 3&  8& Laboratorio 3. Modelos de Regresión Lineal \\
    \hline
    Competencias 2 y 3&  9& Laboratorio 4. Árboles de decisión   \\
    \hline
    Competencias 2 y 3&  10& Laboratorio 5. Bayes ingenuo   \\  
    \hline
    Competencias 2 y 3&  11& Laboratorio 6. KNN  \\ 
    \hline
    Competencias 2 y 3&  12&  Laboratorio 7. Regresión logística\\
    \hline
    Competencias 2 y 3&  13& Avances 2 del proyecto. Investigación de algoritmos de aprendizaje de máquinas que pueden usar. Laboratorio 8. SVM   \\
    \hline
    Competencias 2 y 3&  14& Laboratorio 8. Redes Neuronales  \\
    \hline
    Competencias 2 y 3&  15& Avances 3 del proyecto. Selección de algoritmos a usar. Laboratorio 10. Aprendizaje Semisupervisado   \\  
    \hline
    Competencias 2,3 &  16& Entrega 2 del proyecto. Resultados del proyecto   \\ 
    \hline
    Competencias 2,3 &  17& Laboratorio 11. Aprendizaje por refuerzo  \\
    \hline
    Competencias 1,2,3&  18& Observación y Diálogo participativo  \\
    \hline
    Competencias 1,2,3 &  19& Entrega de Artículo y Poster Científico   \\
    \hline

    \end{tabularx}
    \label{tab:evaluacion}
\end{table}
    
    % ---------------------------------------------------
    % Agregue la bibliografía y enlaces web en la sección de variables.tex
    % ---------------------------------------------------

    \section*{6. Bibliografía y enlaces web}
    \bibliografia
    
    % ---------------------------------------------------

\end{document}
