% variables.tex

% Información general
% Sección de identificación
\newcommand{\facultad}{Facultad de Ingeniería}
\newcommand{\departamento}{Ciencia de la Computación y Tecnologías de la información}
\newcommand{\nombreCurso}{Minería de Datos}
\newcommand{\codigoCurso}{CC3074}
\newcommand{\creditos}{4}
\newcommand{\anio}{2026}
\newcommand{\ciclo}{1}
\newcommand{\teoria}{2}
\newcommand{\practica}{2}
\newcommand{\total}{4}
\newcommand{\horario}{Lunes y Viernes 15:40 - 17:00}
\newcommand{\fechaCreacion}{4/01/2026}
\newcommand{\fechaModificacion}{4/01/2026}

\newcommand{\descripcion}{
La minería de datos y el aprendizaje automático forman parte esencial del análisis moderno de datos, al combinar técnicas estadísticas y métodos de las ciencias de la computación para identificar patrones, relaciones y estructuras relevantes en grandes conjuntos de datos provenientes de diversas fuentes. Estas metodologías permiten extraer información significativa que no es evidente a simple vista y transformarla en conocimiento útil para el análisis, la predicción y la toma de decisiones.

El curso introduce los principios teóricos y prácticos del aprendizaje automático, abordando técnicas para el análisis exploratorio de datos, la clasificación y la predicción mediante modelos supervisados y no supervisados. Se pone especial énfasis en la correcta interpretación de los resultados, la validación de los modelos y la comprensión de sus alcances y limitaciones.

A lo largo del curso, el estudiante trabajará con datos reales y herramientas computacionales actuales, desarrollando la capacidad de seleccionar y aplicar modelos adecuados a distintos problemas en áreas científicas, sociales y tecnológicas. Al finalizar, contará con una formación sólida que le permitirá utilizar técnicas de aprendizaje automático de manera crítica y responsable en diversos contextos profesionales y académicos.
}
\newcommand{\requisitos}{\variosItems[Algoritmos y Programación Básica, Modelos Estadísticos 1, Cálculo 1]}
\newcommand{\modalidades}{\variosItems[Presencial, Virtual sincrónico, Híbrido]}


% Metodologías
% Defina las metodologías activas que se utilizarán en el curso. Para marcar una metodología, 
%ponga una X entre los corchetes antes del texto de la metodología.

\newcommand{\opc}[2][]{\makebox[\linewidth]{#2\hfill(\,#1\,)}}
\newcommand{\metodologiasActivas}{%
  \setlength{\tabcolsep}{8pt}%
  \renewcommand{\arraystretch}{1.2}%
  \noindent\begin{tabularx}{\textwidth}{|X|X|}
    \hline
    \multicolumn{2}{|l|}{\textbf{METODOLOGÍAS ACTIVAS SUGERIDAS (Marque con una X)}} \\ \hline
    % Ponga [X] antes de la llave para marcar la casilla que desee
    %Ejemplo: \opc[X]{Aprendizaje basado en proyectos}
    \opc [X]{Aprendizaje basado en problemas} & \opc [X]{Aprendizaje basado en proyectos} \\ \hline
    \opc{Aprendizaje basado en estudio de casos} & \opc [X]{Aprendizaje basado en investigación} \\ \hline
    \opc{Aprendizaje situado (in situ)} & \opc{Aprendizaje por descubrimiento o heurístico} \\ \hline
    \opc [X]{Autoaprendizaje} & \opc{Aprendizaje entre pares} \\ \hline
    \opc{Aprendizaje basado en clase invertida} & \opc{Aprendizaje basado en retos} \\ \hline
    \opc{Aprendizaje basado en gamificación} & \opc{Aprendizaje basado en design thinking} \\ \hline
    \multicolumn{2}{|p{15cm}|}{\textbf{Otra (especifique):}} \\ \hline

  \end{tabularx}%
}



% Bibliografía
% Tomar en cuenta Bibliografía que se encuentra en la Biblioteca UVG y publicaciones o investigaciones realizadas en UVG. 

\newcommand{\bibliografia}{
    \noindent\setlength{\fboxsep}{10pt}\fbox{%
    \parbox{\textwidth}{%
    \begin{enumerate}[label=\Alph*.]
       \item Canvas: \url{https://uvg.instructure.com} Curso: CC3074
                
                \item Dongre, S. S., \& Malik, L. G. (2017). Rare Class Problem in data mining: Review. \textit{International Journal of Advanced Research in Computer Science}, 8(7), 33–38.
                
                \item Elankavi, R., Kalaiprasath, R., \& Udayakumar, R. (2017, September 15). DATA MINING WITH BIG DATA REVOLUTION HYBRID. \textit{International Journal on Smart Sensing \& Intelligent Systems}.
                
                \item Peng, R. D. (2015). \textit{R Programming for Data Science}. The R Project; R Foundation. \url{https://doi.org/10.1073/pnas.0703993104}
                
                \item Raza, K. (2012). Application of Data mining in Bioinformatics. \textit{Indian Journal of Computer Science and Engineering}, 1(2), 114–118. \url{https://doi.org/10.4010/2016.1759}
                
                \item Sharma, A., \& Kaur, B. (2017, July 15). A RESEARCH REVIEW ON COMPARATIVE ANALYSIS OF DATA MINING TOOLS, TECHNIQUES AND PARAMETERS. \textit{International Journal of Advanced Research in Computer Science}.
                
                \item Gareth, J., Witten, D., Hastie, T., \& Tibshirani, R. (2009). \textit{An Introduction to Statistical Learning with Applications in R}.
                
                \item Hernández Catalán, M. I., \& García Pérez, L. (2016). Predicción de cambio de carrera de los estudiantes de la Universidad del Valle de Guatemala. \textit{Conferencia Internacional de Ingeniería Informática y Sistemas de Información (CIIISI’18)}. La Habana, Cuba. ISBN: 978-959-261-585-4.
                
                \item Mancilla-Cáceres, J. F., Furlán, L., \& García, L. (2019). Exploración de factores asociados al aprendizaje. \textit{Revista de la Universidad del Valle de Guatemala}, (38). \url{https://www.uvg.edu.gt/servicios/volumen-38/}
                
                \item Universidad del Valle de Guatemala. (n.d.). Reglamento UVG-13. \url{https://res.cloudinary.com/webuvg/image/upload/v1541189810/WEB/Nosotros/reglamentos/Reg-uvg-13.pdf}
    \end{enumerate}
     }%
  }%
}

\newcommand{\politicas}{
  \textbf{ Acerca de exámenes parciales, exámenes cortos, comprobaciones de lectura (y cualquier otra evaluación presencial en clase que aplique):}
    \begin{itemize}
      \item Se programarán durante los períodos de clase.
      \item En el caso de exámenes parciales se entregarán hojas a los estudiantes para que, a mano, escriban en ellas las respuestas y soluciones requeridas.
      \item En el caso de exámenes cortos y comprobaciones de lectura, los estudiantes deben contar con hojas de papel para responder a mano las respuestas.
      \item No está permitido que el estudiante se levante de su escritorio o mesa ni que hable durante la realización de este tipo de evaluaciones, salvo que por su naturaleza sean en grupo.
      \item Para todo tipo de exámenes no está permitido que el estudiante cuente con dispositivos electrónicos en su área de trabajo ni dentro de su ropa o en su cuerpo. Esta disposición incluye: teléfonos celulares, relojes inteligentes, audífonos de cualquier tipo, anteojos inteligentes, bocinas, tablets y laptops o cualquier otro dispositivo electrónico a través del cual se pueda consultar información o comunicarse de cualquier forma con otras personas.
      \item En caso se observe que un estudiante tiene un dispositivo electrónico en su área de trabajo, ropa o cuerpo durante la evaluación; se le llamará la atención inmediatamente y la evaluación no se calificará. Para el caso de tablets y laptops, se exceptúan los casos en los que se es indispensable el uso de software, previamente autorizado por el docente.
      \item Cualquier estudiante que sea sorprendido copiando en un examen o presentando un trabajo que no sea propio (entiéndase PLAGIO) será sujeto a las medidas disciplinarias que contemple el reglamento de la Universidad.
      \item Si un estudiante se encuentra inconforme con la nota obtenida en una evaluación, podrá solicitar al docente una revisión dentro de un periodo de 5 días hábiles después de publicada la nota. La evaluación no debe presentar ninguna señal de alteración, y el caso de ser resuelta a mano, debe estar escrita con lapicero sin tachones.
    \end{itemize}
    \textbf{ Acerca del uso de Inteligencia Artificial:}
    \begin{itemize}
      \item Los estudiantes podrán hacer uso de inteligencia artificial, siempre y cuando la utilicen de forma ética y responsable, con juicio crítico, en los casos en que amerita ser utilizada y que sea para contribuir en su aprendizaje y desarrollo de competencias, de acuerdo con los lineamientos que establezca el docente y citando su uso de la forma correcta en sus informes.
      \item En ningún caso podrán presentar como propia información obtenida a través de esta tecnología. Es importante que los estudiantes revisen las recomendaciones al respecto en estos documentos que se encuentran en la página web institucional: 
      \begin{itemize}[label=-]
        \item \href{https://res.cloudinary.com/webuvg/image/upload/v1737393102/WEB/Nosotros/reglamentos/2025/guia-ia-estudiantes.pdf}{Guía para el uso ético de la inteligencia artificial}.
         \item \href{https://res.cloudinary.com/webuvg/image/upload/v1737393102/WEB/Nosotros/reglamentos/2025/politica-ia-uvg.pdf}{Guía rápida para el uso ético y responsable de la inteligencia artificial}.
      \end{itemize}
    \end{itemize}
}