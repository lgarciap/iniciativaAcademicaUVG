% variables.tex

% Información general
% Sección de identificación
\newcommand{\facultad}{Facultad de Ingeniería}
\newcommand{\departamento}{Ciencia de la Computación y Tecnologías de la información}
\newcommand{\nombreCurso}{Nombre Curso}
\newcommand{\codigoCurso}{cc1234}
\newcommand{\creditos}{4}
\newcommand{\anio}{2025}
\newcommand{\ciclo}{2}
\newcommand{\teoria}{2}
\newcommand{\practica}{2}
\newcommand{\total}{2}
\newcommand{\horario}{Lunes y Miércoles 8:00 - 10:00}
\newcommand{\fechaCreacion}{1/7/2024}
\newcommand{\fechaModificacion}{1/06/2025}

\newcommand{\descripcion}{
Descripción del curso
}
\newcommand{\requisitos}{\variosItems[Requisito 1, Requisito 2, Requisito n ]}
\newcommand{\modalidades}{\variosItems[Modalidad 1, Modalidad 2, Modalidad n]}


% Metodologías
% Defina las metodologías activas que se utilizarán en el curso. Para marcar una metodología, 
%ponga una X entre los corchetes antes del texto de la metodología.

\newcommand{\opc}[2][]{\makebox[\linewidth]{#2\hfill(\,#1\,)}}
\newcommand{\metodologiasActivas}{%
  \setlength{\tabcolsep}{8pt}%
  \renewcommand{\arraystretch}{1.2}%
  \noindent\begin{tabularx}{\textwidth}{|X|X|}
    \hline
    \multicolumn{2}{|l|}{\textbf{METODOLOGÍAS ACTIVAS SUGERIDAS (Marque con una X)}} \\ \hline
    % Ponga [X] antes de la llave para marcar la casilla que desee
    %Ejemplo: \opc[X]{Aprendizaje basado en proyectos}
    \opc{Aprendizaje basado en problemas} & \opc{Aprendizaje basado en proyectos} \\ \hline
    \opc{Aprendizaje basado en estudio de casos} & \opc{Aprendizaje basado en investigación} \\ \hline
    \opc{Aprendizaje situado (in situ)} & \opc{Aprendizaje por descubrimiento o heurístico} \\ \hline
    \opc{Autoaprendizaje} & \opc{Aprendizaje entre pares} \\ \hline
    \opc{Aprendizaje basado en clase invertida} & \opc{Aprendizaje basado en retos} \\ \hline
    \opc{Aprendizaje basado en gamificación} & \opc{Aprendizaje basado en design thinking} \\ \hline
    \multicolumn{2}{|p{15cm}|}{\textbf{Otra (especifique):}} \\ \hline

  \end{tabularx}%
}



% Bibliografía
% Tomar en cuenta Bibliografía que se encuentra en la Biblioteca UVG y publicaciones o investigaciones realizadas en UVG. 

\newcommand{\bibliografia}{
    \noindent\setlength{\fboxsep}{10pt}\fbox{%
    \parbox{\textwidth}{%
    \begin{enumerate}[label=\Alph*.]
        \item Canvas: \url{https://uvg.instructure.com} Curso: CC2005
        \item Codecademy. (n.d.). Python course catalog. Codecademy. \url{https://www.codecademy.com/catalog/language/python}
        \item Downey, Allen. \textit{Think Python: How to Think Like a Computer Scientist}. Learning with Python. ISBN 13:9780521898119. \url{http://www.thinkpython.com}. Versión electrónica 3
        \item Gonzáles Duque, Raúl. \textit{Python para todos}. \url{http://mundogeek.net/tutorial-python/}
        \item Rice, John K. \& Rice. John R. \textit{Introduction to Computer Science}. Holt, Rinehart and Winston Inc. SBN: 03-067525-1.
        \item Pandas Development Team. (n.d.). pandas documentation. pandas Documentation. \url{https://pandas.pydata.org/docs/}
        \item Streamlit. (n.d.). Cheat sheet. Streamlit Documentation. \url{https://docs.streamlit.io/develop/quick-reference/cheat-sheet}
        \item Universidad del Valle de Guatemala. (n.d.). Reglamento UVG-13. \url{https://res.cloudinary.com/webuvg/image/upload/v1541189810/WEB/Nosotros/reglamentos/Reg-uvg-13.pdf}
    \end{enumerate}
     }%
  }%
}

