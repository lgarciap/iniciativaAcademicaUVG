% variables.tex

% Información general
% Sección de identificación
\newcommand{\facultad}{Facultad de Ingeniería}
\newcommand{\departamento}{Ciencia de la Computación y Tecnologías de la información}
\newcommand{\nombreCurso}{Nombre Curso}
\newcommand{\codigoCurso}{cc1234}
\newcommand{\creditos}{4}
\newcommand{\anio}{2025}
\newcommand{\ciclo}{2}
\newcommand{\teoria}{2}
\newcommand{\practica}{2}
\newcommand{\total}{2}
\newcommand{\horario}{Lunes y Miércoles 8:00 - 10:00}
\newcommand{\fechaCreacion}{1/7/2024}
\newcommand{\fechaModificacion}{1/06/2025}

\newcommand{\descripcion}{
Descripción del curso
}
\newcommand{\requisitos}{\variosItems[Requisito 1, Requisito 2, Requisito n ]}
\newcommand{\modalidades}{\variosItems[Modalidad 1, Modalidad 2, Modalidad n]}


% Metodologías
% Defina las metodologías activas que se utilizarán en el curso. Para marcar una metodología, 
%ponga una X entre los corchetes antes del texto de la metodología.

\newcommand{\opc}[2][]{\makebox[\linewidth]{#2\hfill(\,#1\,)}}
\newcommand{\metodologiasActivas}{%
  \setlength{\tabcolsep}{8pt}%
  \renewcommand{\arraystretch}{1.2}%
  \noindent\begin{tabularx}{\textwidth}{|X|X|}
    \hline
    \multicolumn{2}{|l|}{\textbf{METODOLOGÍAS ACTIVAS SUGERIDAS (Marque con una X)}} \\ \hline
    % Ponga [X] antes de la llave para marcar la casilla que desee
    %Ejemplo: \opc[X]{Aprendizaje basado en proyectos}
    \opc{Aprendizaje basado en problemas} & \opc{Aprendizaje basado en proyectos} \\ \hline
    \opc{Aprendizaje basado en estudio de casos} & \opc{Aprendizaje basado en investigación} \\ \hline
    \opc{Aprendizaje situado (in situ)} & \opc{Aprendizaje por descubrimiento o heurístico} \\ \hline
    \opc{Autoaprendizaje} & \opc{Aprendizaje entre pares} \\ \hline
    \opc{Aprendizaje basado en clase invertida} & \opc{Aprendizaje basado en retos} \\ \hline
    \opc{Aprendizaje basado en gamificación} & \opc{Aprendizaje basado en design thinking} \\ \hline
    \multicolumn{2}{|p{15cm}|}{\textbf{Otra (especifique):}} \\ \hline

  \end{tabularx}%
}



% Bibliografía
% Tomar en cuenta Bibliografía que se encuentra en la Biblioteca UVG y publicaciones o investigaciones realizadas en UVG. 

\newcommand{\bibliografia}{
    \noindent\setlength{\fboxsep}{10pt}\fbox{%
    \parbox{\textwidth}{%
    \begin{enumerate}[label=\Alph*.]
        \item Canvas: \url{https://uvg.instructure.com} Curso: CC2005
        \item Codecademy. (n.d.). Python course catalog. Codecademy. \url{https://www.codecademy.com/catalog/language/python}
        \item Downey, Allen. \textit{Think Python: How to Think Like a Computer Scientist}. Learning with Python. ISBN 13:9780521898119. \url{http://www.thinkpython.com}. Versión electrónica 3
        \item Gonzáles Duque, Raúl. \textit{Python para todos}. \url{http://mundogeek.net/tutorial-python/}
        \item Rice, John K. \& Rice. John R. \textit{Introduction to Computer Science}. Holt, Rinehart and Winston Inc. SBN: 03-067525-1.
        \item Pandas Development Team. (n.d.). pandas documentation. pandas Documentation. \url{https://pandas.pydata.org/docs/}
        \item Streamlit. (n.d.). Cheat sheet. Streamlit Documentation. \url{https://docs.streamlit.io/develop/quick-reference/cheat-sheet}
        \item Universidad del Valle de Guatemala. (n.d.). Reglamento UVG-13. \url{https://res.cloudinary.com/webuvg/image/upload/v1541189810/WEB/Nosotros/reglamentos/Reg-uvg-13.pdf}
    \end{enumerate}
     }%
  }%
}

\newcommand{\politicas}{
  \textbf{ Acerca de exámenes parciales, exámenes cortos, comprobaciones de lectura (y cual-quier otra evaluación presencial en clase que aplique):}
    \begin{itemize}
      \item Se programarán durante los períodos de clase.
      \item En el caso de exámenes parciales se entregarán hojas a los estudiantes para que, a mano, escriban en ellas las respuestas y soluciones requeridas.
      \item En el caso de exámenes cortos y comprobaciones de lectura, los estudiantes deben contar con hojas de papel para responder a mano las respuestas.
      \item No está permitido que el estudiante se levante de su escritorio o mesa ni que hable durante la realización de este tipo de evaluaciones, salvo que por su naturaleza sean en grupo.
      \item Para todo tipo de exámenes no está permitido que el estudiante cuente con dispositi-vos electrónicos en su área de trabajo ni dentro de su ropa o en su cuerpo. Esta dispo-sición incluye: teléfonos celulares, relojes inteligentes, audífonos de cualquier tipo, an-teojos inteligentes, bocinas, tablets y laptops o cualquier otro dispositivo electrónico a través del cual se pueda consultar información o comunicarse de cualquier forma con otras personas.
      \item En caso se observe que un estudiante tiene un dispositivo electrónico en su área de trabajo, ropa o cuerpo durante la evaluación; se le llamará la atención inmediatamente y la evaluación no se calificará. Para el caso de tablets y laptops, se exceptúan los ca-sos en los que se es indispensable el uso de software, previamente autorizado por el docente.
      \item Cualquier estudiante que sea sorprendido copiando en un examen o presentando un trabajo que no sea propio (entiéndase PLAGIO) será sujeto a las medidas disciplinarias que contemple el reglamento de la Universidad.
      \item Si un estudiante se encuentra inconforme con la nota obtenida en una evaluación, po-drá solicitar al docente una revisión dentro de un periodo de 5 días hábiles después de publicada la nota. La evaluación no debe presentar ninguna señal de alteración, y el caso de ser resuelta a mano, debe estar escrita con lapicero sin tachones.
    \end{itemize}
    \textbf{ Acerca del uso de Inteligencia Artificial:}
    \begin{itemize}
      \item Los estudiantes podrán hacer uso de inteligencia artificial, siempre y cuando la utilicen de forma ética y responsable, con juicio crítico, en los casos en que amerita ser utili-zada y que sea para contribuir en su aprendizaje y desarrollo de competencias, de acuerdo con los lineamientos que establezca el docente y citando su uso de la forma correcta en sus informes.
      \item En ningún caso podrán presentar como propia información obtenida a través de esta tecnología. Es importante que los estudiantes revisen las recomendaciones al res-pecto en estos documentos que se encuentran en la página web institucional: 
      \begin{itemize}[label=-]
        \item \href{https://res.cloudinary.com/webuvg/image/upload/v1737393102/WEB/Nosotros/reglamentos/2025/guia-ia-estudiantes.pdf}{Guía para el uso ético de la inteligencia artificial}.
         \item \href{https://res.cloudinary.com/webuvg/image/upload/v1737393102/WEB/Nosotros/reglamentos/2025/politica-ia-uvg.pdf}{Guía rápida para el uso ético y responsable de la inteligencia artificial}.
      \end{itemize}
    \end{itemize}
}